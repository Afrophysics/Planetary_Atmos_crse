
% Default to the notebook output style

    


% Inherit from the specified cell style.




    
\documentclass[11pt]{article}

    
    
    \usepackage[T1]{fontenc}
    % Nicer default font (+ math font) than Computer Modern for most use cases
    \usepackage{mathpazo}

    % Basic figure setup, for now with no caption control since it's done
    % automatically by Pandoc (which extracts ![](path) syntax from Markdown).
    \usepackage{graphicx}
    % We will generate all images so they have a width \maxwidth. This means
    % that they will get their normal width if they fit onto the page, but
    % are scaled down if they would overflow the margins.
    \makeatletter
    \def\maxwidth{\ifdim\Gin@nat@width>\linewidth\linewidth
    \else\Gin@nat@width\fi}
    \makeatother
    \let\Oldincludegraphics\includegraphics
    % Set max figure width to be 80% of text width, for now hardcoded.
    \renewcommand{\includegraphics}[1]{\Oldincludegraphics[width=.8\maxwidth]{#1}}
    % Ensure that by default, figures have no caption (until we provide a
    % proper Figure object with a Caption API and a way to capture that
    % in the conversion process - todo).
    \usepackage{caption}
    \DeclareCaptionLabelFormat{nolabel}{}
    \captionsetup{labelformat=nolabel}

    \usepackage{adjustbox} % Used to constrain images to a maximum size 
    \usepackage{xcolor} % Allow colors to be defined
    \usepackage{enumerate} % Needed for markdown enumerations to work
    \usepackage{geometry} % Used to adjust the document margins
    \usepackage{amsmath} % Equations
    \usepackage{amssymb} % Equations
    \usepackage{textcomp} % defines textquotesingle
    % Hack from http://tex.stackexchange.com/a/47451/13684:
    \AtBeginDocument{%
        \def\PYZsq{\textquotesingle}% Upright quotes in Pygmentized code
    }
    \usepackage{upquote} % Upright quotes for verbatim code
    \usepackage{eurosym} % defines \euro
    \usepackage[mathletters]{ucs} % Extended unicode (utf-8) support
    \usepackage[utf8x]{inputenc} % Allow utf-8 characters in the tex document
    \usepackage{fancyvrb} % verbatim replacement that allows latex
    \usepackage{grffile} % extends the file name processing of package graphics 
                         % to support a larger range 
    % The hyperref package gives us a pdf with properly built
    % internal navigation ('pdf bookmarks' for the table of contents,
    % internal cross-reference links, web links for URLs, etc.)
    \usepackage{hyperref}
    \usepackage{longtable} % longtable support required by pandoc >1.10
    \usepackage{booktabs}  % table support for pandoc > 1.12.2
    \usepackage[inline]{enumitem} % IRkernel/repr support (it uses the enumerate* environment)
    \usepackage[normalem]{ulem} % ulem is needed to support strikethroughs (\sout)
                                % normalem makes italics be italics, not underlines
    

    
    
    % Colors for the hyperref package
    \definecolor{urlcolor}{rgb}{0,.145,.698}
    \definecolor{linkcolor}{rgb}{.71,0.21,0.01}
    \definecolor{citecolor}{rgb}{.12,.54,.11}

    % ANSI colors
    \definecolor{ansi-black}{HTML}{3E424D}
    \definecolor{ansi-black-intense}{HTML}{282C36}
    \definecolor{ansi-red}{HTML}{E75C58}
    \definecolor{ansi-red-intense}{HTML}{B22B31}
    \definecolor{ansi-green}{HTML}{00A250}
    \definecolor{ansi-green-intense}{HTML}{007427}
    \definecolor{ansi-yellow}{HTML}{DDB62B}
    \definecolor{ansi-yellow-intense}{HTML}{B27D12}
    \definecolor{ansi-blue}{HTML}{208FFB}
    \definecolor{ansi-blue-intense}{HTML}{0065CA}
    \definecolor{ansi-magenta}{HTML}{D160C4}
    \definecolor{ansi-magenta-intense}{HTML}{A03196}
    \definecolor{ansi-cyan}{HTML}{60C6C8}
    \definecolor{ansi-cyan-intense}{HTML}{258F8F}
    \definecolor{ansi-white}{HTML}{C5C1B4}
    \definecolor{ansi-white-intense}{HTML}{A1A6B2}

    % commands and environments needed by pandoc snippets
    % extracted from the output of `pandoc -s`
    \providecommand{\tightlist}{%
      \setlength{\itemsep}{0pt}\setlength{\parskip}{0pt}}
    \DefineVerbatimEnvironment{Highlighting}{Verbatim}{commandchars=\\\{\}}
    % Add ',fontsize=\small' for more characters per line
    \newenvironment{Shaded}{}{}
    \newcommand{\KeywordTok}[1]{\textcolor[rgb]{0.00,0.44,0.13}{\textbf{{#1}}}}
    \newcommand{\DataTypeTok}[1]{\textcolor[rgb]{0.56,0.13,0.00}{{#1}}}
    \newcommand{\DecValTok}[1]{\textcolor[rgb]{0.25,0.63,0.44}{{#1}}}
    \newcommand{\BaseNTok}[1]{\textcolor[rgb]{0.25,0.63,0.44}{{#1}}}
    \newcommand{\FloatTok}[1]{\textcolor[rgb]{0.25,0.63,0.44}{{#1}}}
    \newcommand{\CharTok}[1]{\textcolor[rgb]{0.25,0.44,0.63}{{#1}}}
    \newcommand{\StringTok}[1]{\textcolor[rgb]{0.25,0.44,0.63}{{#1}}}
    \newcommand{\CommentTok}[1]{\textcolor[rgb]{0.38,0.63,0.69}{\textit{{#1}}}}
    \newcommand{\OtherTok}[1]{\textcolor[rgb]{0.00,0.44,0.13}{{#1}}}
    \newcommand{\AlertTok}[1]{\textcolor[rgb]{1.00,0.00,0.00}{\textbf{{#1}}}}
    \newcommand{\FunctionTok}[1]{\textcolor[rgb]{0.02,0.16,0.49}{{#1}}}
    \newcommand{\RegionMarkerTok}[1]{{#1}}
    \newcommand{\ErrorTok}[1]{\textcolor[rgb]{1.00,0.00,0.00}{\textbf{{#1}}}}
    \newcommand{\NormalTok}[1]{{#1}}
    
    % Additional commands for more recent versions of Pandoc
    \newcommand{\ConstantTok}[1]{\textcolor[rgb]{0.53,0.00,0.00}{{#1}}}
    \newcommand{\SpecialCharTok}[1]{\textcolor[rgb]{0.25,0.44,0.63}{{#1}}}
    \newcommand{\VerbatimStringTok}[1]{\textcolor[rgb]{0.25,0.44,0.63}{{#1}}}
    \newcommand{\SpecialStringTok}[1]{\textcolor[rgb]{0.73,0.40,0.53}{{#1}}}
    \newcommand{\ImportTok}[1]{{#1}}
    \newcommand{\DocumentationTok}[1]{\textcolor[rgb]{0.73,0.13,0.13}{\textit{{#1}}}}
    \newcommand{\AnnotationTok}[1]{\textcolor[rgb]{0.38,0.63,0.69}{\textbf{\textit{{#1}}}}}
    \newcommand{\CommentVarTok}[1]{\textcolor[rgb]{0.38,0.63,0.69}{\textbf{\textit{{#1}}}}}
    \newcommand{\VariableTok}[1]{\textcolor[rgb]{0.10,0.09,0.49}{{#1}}}
    \newcommand{\ControlFlowTok}[1]{\textcolor[rgb]{0.00,0.44,0.13}{\textbf{{#1}}}}
    \newcommand{\OperatorTok}[1]{\textcolor[rgb]{0.40,0.40,0.40}{{#1}}}
    \newcommand{\BuiltInTok}[1]{{#1}}
    \newcommand{\ExtensionTok}[1]{{#1}}
    \newcommand{\PreprocessorTok}[1]{\textcolor[rgb]{0.74,0.48,0.00}{{#1}}}
    \newcommand{\AttributeTok}[1]{\textcolor[rgb]{0.49,0.56,0.16}{{#1}}}
    \newcommand{\InformationTok}[1]{\textcolor[rgb]{0.38,0.63,0.69}{\textbf{\textit{{#1}}}}}
    \newcommand{\WarningTok}[1]{\textcolor[rgb]{0.38,0.63,0.69}{\textbf{\textit{{#1}}}}}
    
    
    % Define a nice break command that doesn't care if a line doesn't already
    % exist.
    \def\br{\hspace*{\fill} \\* }
    % Math Jax compatability definitions
    \def\gt{>}
    \def\lt{<}
    % Document parameters
    \title{HW\_3}
    
    
    

    % Pygments definitions
    
\makeatletter
\def\PY@reset{\let\PY@it=\relax \let\PY@bf=\relax%
    \let\PY@ul=\relax \let\PY@tc=\relax%
    \let\PY@bc=\relax \let\PY@ff=\relax}
\def\PY@tok#1{\csname PY@tok@#1\endcsname}
\def\PY@toks#1+{\ifx\relax#1\empty\else%
    \PY@tok{#1}\expandafter\PY@toks\fi}
\def\PY@do#1{\PY@bc{\PY@tc{\PY@ul{%
    \PY@it{\PY@bf{\PY@ff{#1}}}}}}}
\def\PY#1#2{\PY@reset\PY@toks#1+\relax+\PY@do{#2}}

\expandafter\def\csname PY@tok@w\endcsname{\def\PY@tc##1{\textcolor[rgb]{0.73,0.73,0.73}{##1}}}
\expandafter\def\csname PY@tok@c\endcsname{\let\PY@it=\textit\def\PY@tc##1{\textcolor[rgb]{0.25,0.50,0.50}{##1}}}
\expandafter\def\csname PY@tok@cp\endcsname{\def\PY@tc##1{\textcolor[rgb]{0.74,0.48,0.00}{##1}}}
\expandafter\def\csname PY@tok@k\endcsname{\let\PY@bf=\textbf\def\PY@tc##1{\textcolor[rgb]{0.00,0.50,0.00}{##1}}}
\expandafter\def\csname PY@tok@kp\endcsname{\def\PY@tc##1{\textcolor[rgb]{0.00,0.50,0.00}{##1}}}
\expandafter\def\csname PY@tok@kt\endcsname{\def\PY@tc##1{\textcolor[rgb]{0.69,0.00,0.25}{##1}}}
\expandafter\def\csname PY@tok@o\endcsname{\def\PY@tc##1{\textcolor[rgb]{0.40,0.40,0.40}{##1}}}
\expandafter\def\csname PY@tok@ow\endcsname{\let\PY@bf=\textbf\def\PY@tc##1{\textcolor[rgb]{0.67,0.13,1.00}{##1}}}
\expandafter\def\csname PY@tok@nb\endcsname{\def\PY@tc##1{\textcolor[rgb]{0.00,0.50,0.00}{##1}}}
\expandafter\def\csname PY@tok@nf\endcsname{\def\PY@tc##1{\textcolor[rgb]{0.00,0.00,1.00}{##1}}}
\expandafter\def\csname PY@tok@nc\endcsname{\let\PY@bf=\textbf\def\PY@tc##1{\textcolor[rgb]{0.00,0.00,1.00}{##1}}}
\expandafter\def\csname PY@tok@nn\endcsname{\let\PY@bf=\textbf\def\PY@tc##1{\textcolor[rgb]{0.00,0.00,1.00}{##1}}}
\expandafter\def\csname PY@tok@ne\endcsname{\let\PY@bf=\textbf\def\PY@tc##1{\textcolor[rgb]{0.82,0.25,0.23}{##1}}}
\expandafter\def\csname PY@tok@nv\endcsname{\def\PY@tc##1{\textcolor[rgb]{0.10,0.09,0.49}{##1}}}
\expandafter\def\csname PY@tok@no\endcsname{\def\PY@tc##1{\textcolor[rgb]{0.53,0.00,0.00}{##1}}}
\expandafter\def\csname PY@tok@nl\endcsname{\def\PY@tc##1{\textcolor[rgb]{0.63,0.63,0.00}{##1}}}
\expandafter\def\csname PY@tok@ni\endcsname{\let\PY@bf=\textbf\def\PY@tc##1{\textcolor[rgb]{0.60,0.60,0.60}{##1}}}
\expandafter\def\csname PY@tok@na\endcsname{\def\PY@tc##1{\textcolor[rgb]{0.49,0.56,0.16}{##1}}}
\expandafter\def\csname PY@tok@nt\endcsname{\let\PY@bf=\textbf\def\PY@tc##1{\textcolor[rgb]{0.00,0.50,0.00}{##1}}}
\expandafter\def\csname PY@tok@nd\endcsname{\def\PY@tc##1{\textcolor[rgb]{0.67,0.13,1.00}{##1}}}
\expandafter\def\csname PY@tok@s\endcsname{\def\PY@tc##1{\textcolor[rgb]{0.73,0.13,0.13}{##1}}}
\expandafter\def\csname PY@tok@sd\endcsname{\let\PY@it=\textit\def\PY@tc##1{\textcolor[rgb]{0.73,0.13,0.13}{##1}}}
\expandafter\def\csname PY@tok@si\endcsname{\let\PY@bf=\textbf\def\PY@tc##1{\textcolor[rgb]{0.73,0.40,0.53}{##1}}}
\expandafter\def\csname PY@tok@se\endcsname{\let\PY@bf=\textbf\def\PY@tc##1{\textcolor[rgb]{0.73,0.40,0.13}{##1}}}
\expandafter\def\csname PY@tok@sr\endcsname{\def\PY@tc##1{\textcolor[rgb]{0.73,0.40,0.53}{##1}}}
\expandafter\def\csname PY@tok@ss\endcsname{\def\PY@tc##1{\textcolor[rgb]{0.10,0.09,0.49}{##1}}}
\expandafter\def\csname PY@tok@sx\endcsname{\def\PY@tc##1{\textcolor[rgb]{0.00,0.50,0.00}{##1}}}
\expandafter\def\csname PY@tok@m\endcsname{\def\PY@tc##1{\textcolor[rgb]{0.40,0.40,0.40}{##1}}}
\expandafter\def\csname PY@tok@gh\endcsname{\let\PY@bf=\textbf\def\PY@tc##1{\textcolor[rgb]{0.00,0.00,0.50}{##1}}}
\expandafter\def\csname PY@tok@gu\endcsname{\let\PY@bf=\textbf\def\PY@tc##1{\textcolor[rgb]{0.50,0.00,0.50}{##1}}}
\expandafter\def\csname PY@tok@gd\endcsname{\def\PY@tc##1{\textcolor[rgb]{0.63,0.00,0.00}{##1}}}
\expandafter\def\csname PY@tok@gi\endcsname{\def\PY@tc##1{\textcolor[rgb]{0.00,0.63,0.00}{##1}}}
\expandafter\def\csname PY@tok@gr\endcsname{\def\PY@tc##1{\textcolor[rgb]{1.00,0.00,0.00}{##1}}}
\expandafter\def\csname PY@tok@ge\endcsname{\let\PY@it=\textit}
\expandafter\def\csname PY@tok@gs\endcsname{\let\PY@bf=\textbf}
\expandafter\def\csname PY@tok@gp\endcsname{\let\PY@bf=\textbf\def\PY@tc##1{\textcolor[rgb]{0.00,0.00,0.50}{##1}}}
\expandafter\def\csname PY@tok@go\endcsname{\def\PY@tc##1{\textcolor[rgb]{0.53,0.53,0.53}{##1}}}
\expandafter\def\csname PY@tok@gt\endcsname{\def\PY@tc##1{\textcolor[rgb]{0.00,0.27,0.87}{##1}}}
\expandafter\def\csname PY@tok@err\endcsname{\def\PY@bc##1{\setlength{\fboxsep}{0pt}\fcolorbox[rgb]{1.00,0.00,0.00}{1,1,1}{\strut ##1}}}
\expandafter\def\csname PY@tok@kc\endcsname{\let\PY@bf=\textbf\def\PY@tc##1{\textcolor[rgb]{0.00,0.50,0.00}{##1}}}
\expandafter\def\csname PY@tok@kd\endcsname{\let\PY@bf=\textbf\def\PY@tc##1{\textcolor[rgb]{0.00,0.50,0.00}{##1}}}
\expandafter\def\csname PY@tok@kn\endcsname{\let\PY@bf=\textbf\def\PY@tc##1{\textcolor[rgb]{0.00,0.50,0.00}{##1}}}
\expandafter\def\csname PY@tok@kr\endcsname{\let\PY@bf=\textbf\def\PY@tc##1{\textcolor[rgb]{0.00,0.50,0.00}{##1}}}
\expandafter\def\csname PY@tok@bp\endcsname{\def\PY@tc##1{\textcolor[rgb]{0.00,0.50,0.00}{##1}}}
\expandafter\def\csname PY@tok@fm\endcsname{\def\PY@tc##1{\textcolor[rgb]{0.00,0.00,1.00}{##1}}}
\expandafter\def\csname PY@tok@vc\endcsname{\def\PY@tc##1{\textcolor[rgb]{0.10,0.09,0.49}{##1}}}
\expandafter\def\csname PY@tok@vg\endcsname{\def\PY@tc##1{\textcolor[rgb]{0.10,0.09,0.49}{##1}}}
\expandafter\def\csname PY@tok@vi\endcsname{\def\PY@tc##1{\textcolor[rgb]{0.10,0.09,0.49}{##1}}}
\expandafter\def\csname PY@tok@vm\endcsname{\def\PY@tc##1{\textcolor[rgb]{0.10,0.09,0.49}{##1}}}
\expandafter\def\csname PY@tok@sa\endcsname{\def\PY@tc##1{\textcolor[rgb]{0.73,0.13,0.13}{##1}}}
\expandafter\def\csname PY@tok@sb\endcsname{\def\PY@tc##1{\textcolor[rgb]{0.73,0.13,0.13}{##1}}}
\expandafter\def\csname PY@tok@sc\endcsname{\def\PY@tc##1{\textcolor[rgb]{0.73,0.13,0.13}{##1}}}
\expandafter\def\csname PY@tok@dl\endcsname{\def\PY@tc##1{\textcolor[rgb]{0.73,0.13,0.13}{##1}}}
\expandafter\def\csname PY@tok@s2\endcsname{\def\PY@tc##1{\textcolor[rgb]{0.73,0.13,0.13}{##1}}}
\expandafter\def\csname PY@tok@sh\endcsname{\def\PY@tc##1{\textcolor[rgb]{0.73,0.13,0.13}{##1}}}
\expandafter\def\csname PY@tok@s1\endcsname{\def\PY@tc##1{\textcolor[rgb]{0.73,0.13,0.13}{##1}}}
\expandafter\def\csname PY@tok@mb\endcsname{\def\PY@tc##1{\textcolor[rgb]{0.40,0.40,0.40}{##1}}}
\expandafter\def\csname PY@tok@mf\endcsname{\def\PY@tc##1{\textcolor[rgb]{0.40,0.40,0.40}{##1}}}
\expandafter\def\csname PY@tok@mh\endcsname{\def\PY@tc##1{\textcolor[rgb]{0.40,0.40,0.40}{##1}}}
\expandafter\def\csname PY@tok@mi\endcsname{\def\PY@tc##1{\textcolor[rgb]{0.40,0.40,0.40}{##1}}}
\expandafter\def\csname PY@tok@il\endcsname{\def\PY@tc##1{\textcolor[rgb]{0.40,0.40,0.40}{##1}}}
\expandafter\def\csname PY@tok@mo\endcsname{\def\PY@tc##1{\textcolor[rgb]{0.40,0.40,0.40}{##1}}}
\expandafter\def\csname PY@tok@ch\endcsname{\let\PY@it=\textit\def\PY@tc##1{\textcolor[rgb]{0.25,0.50,0.50}{##1}}}
\expandafter\def\csname PY@tok@cm\endcsname{\let\PY@it=\textit\def\PY@tc##1{\textcolor[rgb]{0.25,0.50,0.50}{##1}}}
\expandafter\def\csname PY@tok@cpf\endcsname{\let\PY@it=\textit\def\PY@tc##1{\textcolor[rgb]{0.25,0.50,0.50}{##1}}}
\expandafter\def\csname PY@tok@c1\endcsname{\let\PY@it=\textit\def\PY@tc##1{\textcolor[rgb]{0.25,0.50,0.50}{##1}}}
\expandafter\def\csname PY@tok@cs\endcsname{\let\PY@it=\textit\def\PY@tc##1{\textcolor[rgb]{0.25,0.50,0.50}{##1}}}

\def\PYZbs{\char`\\}
\def\PYZus{\char`\_}
\def\PYZob{\char`\{}
\def\PYZcb{\char`\}}
\def\PYZca{\char`\^}
\def\PYZam{\char`\&}
\def\PYZlt{\char`\<}
\def\PYZgt{\char`\>}
\def\PYZsh{\char`\#}
\def\PYZpc{\char`\%}
\def\PYZdl{\char`\$}
\def\PYZhy{\char`\-}
\def\PYZsq{\char`\'}
\def\PYZdq{\char`\"}
\def\PYZti{\char`\~}
% for compatibility with earlier versions
\def\PYZat{@}
\def\PYZlb{[}
\def\PYZrb{]}
\makeatother


    % Exact colors from NB
    \definecolor{incolor}{rgb}{0.0, 0.0, 0.5}
    \definecolor{outcolor}{rgb}{0.545, 0.0, 0.0}



    
    % Prevent overflowing lines due to hard-to-break entities
    \sloppy 
    % Setup hyperref package
    \hypersetup{
      breaklinks=true,  % so long urls are correctly broken across lines
      colorlinks=true,
      urlcolor=urlcolor,
      linkcolor=linkcolor,
      citecolor=citecolor,
      }
    % Slightly bigger margins than the latex defaults
    
    \geometry{verbose,tmargin=1in,bmargin=1in,lmargin=1in,rmargin=1in}
    
    

    \begin{document}
    
    
    \maketitle
    
    

    
    \section{Earth 164 Planetary Atmospheres HW
3}\label{earth-164-planetary-atmospheres-hw-3}

\subsubsection{Jesus Javier Serrano}\label{jesus-javier-serrano}

\paragraph{2/6/19}\label{section}

\subsection{Bake the atmosphere from
bottom}\label{bake-the-atmosphere-from-bottom}

Suppose an observation is being made on the planetary atmosphere of a
gas giant whose main source of heat energy is from the internal
radiation flux coming from the surface of the planet. The atmosphere is
composed of Hydrogen gas which is a diatomic molecule with the adiabatic
index being \(\gamma = \frac{7}{5}\) and a collision-induced absorption
opacity of \(\tau = A p^2\) where \(A\) is a constant and \(p\) is
pressure. An intrinsic flux \(F\) seeps out of the surface and radiates
towards the atmosphere. In order to calculate the temperature profile of
the air parcel within the atmosphere we must first apply a two-flow
model. This model states that for any incoming intensity \(I\) that
interacts with the atmosphere, a reflecting \(I^{-}\) and transmitting
ray \(I^{+}\) is radiated from the interaction where
\(I = I^{+} + I^{-}\). We shall also assume that the atmosphere is in
monochromative equilibrium so that
\(\frac{\partial F}{\partial z} = 0\). A blackbody distribution was
applied to the atmosphere in order to state that for any radiation that
is absorbed from the atmosphere, it is radiated out again as blackbody
radiation \(B_{\nu}\). Thus, focusing in a single frequency of light:
\[I^{+}_{\nu} = B_{\nu}(T) + \frac{1}{2\pi}F_{\nu}\]
\[I^{-}_{\nu} = B_{\nu}(T) - \frac{1}{2\pi}F_{\nu}\] Recall that for any
radiative transfer within a system, the following equation must hold
true:
\[\mu \frac{\partial I_{\nu}}{\partial \tau_{\nu}} = - I_{\nu} + S_{\nu}\]
where in this case our only source of radiation is the planet's surface
which acts as a blackbody, so \(S_{\nu} = B_{\nu}\). In this image we
will focus in three areas of interests: \(S_1, S_2, S_3\). In \(S_1\)
the surface emits energy with intensity \(I_g\). The energy flux is
scattered into two flux beams, \(I_g^+\) emitting upward and \(I_g^-\)
emitting downward. Both intensities are absorbed and remitted as part of
a blackbody spectrum \(V_{\nu}\) within the atmosphere resulting in a
net flux \(F\) emitting upward. At the top of the atmosphere there is no
reflection and only scattering and emission contributes to the net flux
flowing outwards.

Thus we are able to define the total radiative flux as a function of
opacity if we are able to solve the differential equation above. Before
we try to solve the equaiton, we must first define the constraints of
the system. In order for the greenhouse effect to be true within this
gas planet, two conditions must apply: 1. The radiative transfer at the
surface of the planet must result in an upward intensity of
\(I^{+}_{\nu g} = B_{\nu}(T_1) + \frac{1}{2\pi}F_{\nu} = B_{\nu}(T_g)\).
2. At the top of the atmosphere the net flux must result from upward
going fluxes which means that \(I^{-}_{\nu 0} = 0\) 3.
\(\frac{\partial F}{\partial \tau} = 0\) with no gradient on the
magnitude of the flux along the height of the atmosphere

With these conditions we are now able to define a singular solution for
the temperature profile of the atmosphere. First, we know that the net
normal flux from radiation as a function of intensity and the
infinitesimal scattering solid angle \(d\Omega\) is:
\[F = \int I \cos(\theta) d\Omega= \int I \cos(\theta) \sin(\theta) d\theta d\psi\]
For simplicity a substitution is defined such that
\[\mu = \cos(\theta)\] and \$d \mu = -\sin(\theta) d\theta \$ so:
\[F = \int_{0}^{2\pi} \int_{0}^{\pi} I \cos(\theta) \sin(\theta) d\theta d\psi = \int_{0}^{2\pi} \int_{-1}^{1} I \mu d\mu d\psi.\]
Now let us use Eddington's method of solving this equation and determine
the solution. First we may apply a first order approximation to the
scattered intensity so that: \[I = I_o + \mu I_1\] where I\_o and I\_1
are seperate expansion terms that are independent of \(\mu\). The
radiative flux can now be defined as such:
\[F = \int_{0}^{2\pi} \int_{-1}^{1}  \mu(I_o + \mu I_1 )d\mu d\psi = 2 \pi * (\frac{2 I_1}{3})\]
The integral wasn't that hard and we can actually take advantage of the
fact that \(\mu\) is an orthonormal function being integrated within a
period. Thus, let us define the solid angle integral as a weighting
function such that:
\[<1> = \int_{0}^{2\pi} \int_{-1}^{1} d\mu d\psi = 4 \pi\]
\[<I> = \int_{0}^{2\pi} \int_{-1}^{1} (I_o + \mu I_1 )d\mu d\psi = 4 \pi I_o\]
\[<\mu^1 I> = \int_{0}^{2\pi} \int_{-1}^{1} \mu(I_o + \mu I_1 )d\mu d\psi = \frac{4 \pi I_1}{3}= F\]
\[<\mu^2 I> = \int_{0}^{2\pi} \int_{-1}^{1} \mu^2(I_o + \mu I_1 )d\mu d\psi = \frac{4 \pi I_o}{3}\]
Now we can apply calculus and build a system of equations that will
result in a nice equation. Then: 1.
\[<\mu \frac{\partial I_{\nu}}{\partial \tau_{\nu}}> = <- I_{\nu} + B_{\nu}>\]
2.
\[<\mu^2 \frac{\partial I_{\nu}}{\partial \tau_{\nu}}> = <- \mu I_{\nu} + \mu B_{\nu}>\]
Integrating equaiton 1 results in the following:
\[<\mu \frac{\partial I_{\nu}}{\partial \tau_{\nu}}> = \frac{\partial <\mu I_{\nu}>}{\partial \tau_{\nu}} = \frac{\partial F}{\partial \tau_{\nu}}  = - 4 \pi I_{\nu o} +4\pi B_{\nu}\]
Now integrating equation 2 results in:
\[<\mu^2 \frac{\partial I_{\nu}}{\partial \tau_{\nu}}> = \frac{\partial <\mu^2 I_{\nu}>}{\partial \tau_{\nu}} = \frac{4 \pi}{3}\frac{\partial I_{\nu o}}{\partial \tau_{\nu}}= -F\]
The flux \(F\) is a result from integrating the odd orthonormal function
applied to \(B_{\nu}\) thus making it zero and only focusing on the
intensity. Now that we have our two equations we can now apply
substitution. For the first modified equation we shall look into the
derivative relative to the opacity. Thus:
\[\frac{\partial^2 F}{\partial \tau_{\nu}^2}  = - 4 \pi \frac{\partial I_{\nu o}}{\partial \tau_{\nu}} +4\pi \frac{\partial B_{\nu }}{\partial \tau_{\nu}} \]
From equation 2 we can substitute \(I_{\nu o }\) with \(F\) and have:
\[\frac{\partial^2 F}{\partial \tau_{\nu}^2}  = 3F +4\pi \frac{\partial B_{\nu }}{\partial \tau_{\nu}} \]
In this case we know that
\(\frac{\partial^2 F}{\partial \tau_{\nu}^2} = 0\) and from the second
constraint of the atmosphere \(F = 2 \pi B_{\nu}(T_o)\). Thus:
\[\frac{3}{2} B_{\nu}(T_o) = \frac{\partial B_{\nu }}{\partial \tau_{\nu}}\]
\[\frac{3}{2} B_{\nu}(T_o) d \tau = d B_{\nu}\] Integrate both sides
within the interval of optical depth where at the top-edge of the
atmosphere \(\tau = 0\) and the temperature is set at \(T_o\):
\[\int_{0}^{\tau}\frac{3}{2} B_{\nu}(T_o) d \tau = \int_{B_{\nu}(T_o)}^{B_{\nu}(\tau)} d B_{\nu}\]
\[\frac{3}{2} B_{\nu}(T_o) \tau = B_{\nu}(\tau) - B_{\nu}(T_o)\]
\[ B_{\nu}(T_o)(1 + \frac{3}{2} \tau) = B_{\nu}(\tau)\] Now integrate
the entire equation over all of velocity space so that
\(\int_{0}^{\infty}B_{\nu}d\nu = \sigma T^4\) to get:
\[ \sigma T_o^4(1 + \frac{3}{2} \tau) = \sigma T^4(\tau)\] The final
relation is found such that the equilibrium radiative temperature as a
function of optical depth is:
\[ T_o^4(1 + \frac{3}{2} \tau) = T^4(\tau)\] Temperature as a function
of pressure is: \[ T_o^4(1 + \frac{3}{2} Ap^2) = T^4(p)\]

    \begin{Verbatim}[commandchars=\\\{\}]
{\color{incolor}In [{\color{incolor}1}]:} \PY{k+kn}{import} \PY{n+nn}{numpy} \PY{k}{as} \PY{n+nn}{np}
        \PY{k+kn}{import} \PY{n+nn}{matplotlib}\PY{n+nn}{.}\PY{n+nn}{pyplot} \PY{k}{as} \PY{n+nn}{plt}
        
        
        
        \PY{c+c1}{\PYZsh{}Let us define the radiative equilibrium temperature as a function of p:}
        \PY{k}{def} \PY{n+nf}{Temp}\PY{p}{(}\PY{n}{T\PYZus{}0}\PY{p}{,} \PY{n}{A}\PY{p}{,} \PY{n}{p}\PY{p}{)}\PY{p}{:}
            \PY{k}{return} \PY{n}{T\PYZus{}0}\PY{o}{*}\PY{p}{(}\PY{p}{(}\PY{l+m+mi}{1}\PY{o}{+}\PY{p}{(}\PY{l+m+mf}{1.5}\PY{o}{*}\PY{n}{A}\PY{o}{*}\PY{p}{(}\PY{n}{p}\PY{o}{*}\PY{o}{*}\PY{l+m+mi}{2}\PY{p}{)}\PY{p}{)}\PY{p}{)}\PY{o}{*}\PY{o}{*}\PY{p}{(}\PY{l+m+mf}{0.25}\PY{p}{)}\PY{p}{)}
        
        \PY{n}{interval\PYZus{}pressure} \PY{o}{=} \PY{n}{np}\PY{o}{.}\PY{n}{linspace}\PY{p}{(}\PY{l+m+mi}{100}\PY{p}{,} \PY{l+m+mi}{0}\PY{p}{,} \PY{l+m+mi}{20}\PY{p}{)}
        \PY{n}{Temp\PYZus{}0} \PY{o}{=} \PY{l+m+mf}{2.5} \PY{c+c1}{\PYZsh{}mean temperature of the CMB}
        \PY{n}{A\PYZus{}guess} \PY{o}{=} \PY{l+m+mf}{0.2}
        \PY{n}{result\PYZus{}temp} \PY{o}{=} \PY{n}{Temp}\PY{p}{(}\PY{n}{np}\PY{o}{.}\PY{n}{repeat}\PY{p}{(}\PY{n}{Temp\PYZus{}0}\PY{p}{,} \PY{n+nb}{len}\PY{p}{(}\PY{n}{interval\PYZus{}pressure}\PY{p}{)}\PY{p}{)}\PY{p}{,} \PY{n}{np}\PY{o}{.}\PY{n}{repeat}\PY{p}{(}\PY{n}{A\PYZus{}guess}\PY{p}{,} \PY{n+nb}{len}\PY{p}{(}\PY{n}{interval\PYZus{}pressure}\PY{p}{)}\PY{p}{)}\PY{p}{,} \PY{n}{interval\PYZus{}pressure}\PY{p}{)}
        \PY{n}{plt}\PY{o}{.}\PY{n}{plot}\PY{p}{(}\PY{n}{result\PYZus{}temp}\PY{p}{,} \PY{n}{interval\PYZus{}pressure}\PY{p}{)}
        \PY{n}{plt}\PY{o}{.}\PY{n}{ylabel}\PY{p}{(}\PY{l+s+sa}{r}\PY{l+s+s1}{\PYZsq{}}\PY{l+s+s1}{Units of pressure \PYZdl{}p\PYZdl{}}\PY{l+s+s1}{\PYZsq{}}\PY{p}{)}
        \PY{n}{plt}\PY{o}{.}\PY{n}{xlabel}\PY{p}{(}\PY{l+s+sa}{r}\PY{l+s+s1}{\PYZsq{}}\PY{l+s+s1}{Units of Temperature \PYZdl{}T\PYZus{}}\PY{l+s+si}{\PYZob{}eq\PYZcb{}}\PY{l+s+s1}{\PYZdl{}}\PY{l+s+s1}{\PYZsq{}}\PY{p}{)}
        \PY{n}{plt}\PY{o}{.}\PY{n}{ylim}\PY{p}{(}\PY{l+m+mi}{100}\PY{p}{,} \PY{l+m+mi}{0}\PY{p}{)}
        \PY{n}{plt}\PY{o}{.}\PY{n}{title}\PY{p}{(}\PY{l+s+s1}{\PYZsq{}}\PY{l+s+s1}{Radiative Equilibrium Temperature as a function of pressure}\PY{l+s+s1}{\PYZsq{}}\PY{p}{)}
        \PY{n}{plt}\PY{o}{.}\PY{n}{grid}\PY{p}{(}\PY{k+kc}{True}\PY{p}{)}
        \PY{n}{plt}\PY{o}{.}\PY{n}{show}\PY{p}{(}\PY{p}{)}
        
        \PY{c+c1}{\PYZsh{}}
\end{Verbatim}

    \begin{center}
    \adjustimage{max size={0.9\linewidth}{0.9\paperheight}}{output_1_0.png}
    \end{center}
    { \hspace*{\fill} \\}
    
    Let us analyze the potential temperature profile in order to get a
cleaner picture of the adiabatic process of the air-parcel in this
atmosphere. From the third law of thermodynamics we know that:
\(TdS = C_p dT - VdP\). For an ideal gas we can modify the equation by
stating the fact that \(V = \frac{nRT}{P}\) at constant temperature.
Then: \[dS = C_p (\frac{dT}{T} - \frac{nRdP}{C_p P})\] For the ideal
case of one mole of Hydrogen gas in the atmosphere we further simplify
the equation by noting the following: If \(g(t) = \ln(x(t)y(t))\) then
\[d(g(t)) = \frac{x'(t) y(t) + x(t)y'(t)}{x(t)y(t)} = \frac{dx}{x} + \frac{dy}{y}\]
So: \[\frac{dT}{T} - \frac{nRdP}{C_p P} = d(\ln(T P^{\frac{-R}{C_p}}))\]
Thus: \[dS = C_pd(\ln(T P^{\frac{-R}{C_p}}))\]
\[\frac{\triangle S}{C_p} = \ln \left(\frac{T P^{\frac{-R}{C_p}}}{T_o P_o^{\frac{-R}{C_p}}}\right)\]
\[T_o e^{\frac{\triangle S}{C_p}} = T \left(\frac{P}{P_o}\right)^{\frac{-R}{C_p}} = \theta(T)\]
Notice that if entropy is held constant within the atmosphere then
\(\theta(T) = T_o\). In terms of pressure this equation is defined as:
\[\theta(P) =  T_o(1 + \frac{3}{2} AP^2)^{1/4} \left(\frac{P}{P_o}\right)^{\frac{-R}{C_p}}\]
Since the atmosphere consists of mainly diatomic Hydrogen then
\(C_p = \frac{7}{2}R\) and:
\[\theta(P) =  T_o(1 + \frac{3}{2} AP^2)^{1/4} \left(\frac{P}{P_o}\right)^{\frac{-2}{7}}\]

    \begin{Verbatim}[commandchars=\\\{\}]
{\color{incolor}In [{\color{incolor}3}]:} \PY{k+kn}{import} \PY{n+nn}{numpy} \PY{k}{as} \PY{n+nn}{np}
        \PY{k+kn}{import} \PY{n+nn}{matplotlib}\PY{n+nn}{.}\PY{n+nn}{pyplot} \PY{k}{as} \PY{n+nn}{plt}
        
        
        
        \PY{c+c1}{\PYZsh{}Let us define the radiative equilibrium temperature as a function of p:}
        \PY{k}{def} \PY{n+nf}{Temp}\PY{p}{(}\PY{n}{T\PYZus{}0}\PY{p}{,} \PY{n}{A}\PY{p}{,} \PY{n}{p}\PY{p}{)}\PY{p}{:}
            \PY{k}{return} \PY{n}{T\PYZus{}0}\PY{o}{*}\PY{p}{(}\PY{p}{(}\PY{l+m+mi}{1}\PY{o}{+}\PY{p}{(}\PY{l+m+mf}{1.5}\PY{o}{*}\PY{n}{A}\PY{o}{*}\PY{p}{(}\PY{n}{p}\PY{o}{*}\PY{o}{*}\PY{l+m+mi}{2}\PY{p}{)}\PY{p}{)}\PY{p}{)}\PY{o}{*}\PY{o}{*}\PY{p}{(}\PY{l+m+mf}{0.25}\PY{p}{)}\PY{p}{)}
        
        \PY{k}{def} \PY{n+nf}{pot\PYZus{}Temp}\PY{p}{(}\PY{n}{T\PYZus{}0}\PY{p}{,} \PY{n}{A}\PY{p}{,} \PY{n}{p}\PY{p}{,} \PY{n}{p\PYZus{}o}\PY{p}{)}\PY{p}{:}
            \PY{k}{return} \PY{n}{T\PYZus{}0}\PY{o}{*}\PY{p}{(}\PY{p}{(}\PY{l+m+mi}{1}\PY{o}{+}\PY{p}{(}\PY{l+m+mf}{1.5}\PY{o}{*}\PY{n}{A}\PY{o}{*}\PY{p}{(}\PY{n}{p}\PY{o}{*}\PY{o}{*}\PY{l+m+mi}{2}\PY{p}{)}\PY{p}{)}\PY{p}{)}\PY{o}{*}\PY{o}{*}\PY{p}{(}\PY{l+m+mf}{0.25}\PY{p}{)}\PY{p}{)}\PY{o}{*}\PY{p}{(}\PY{p}{(}\PY{n}{p}\PY{o}{/}\PY{n}{p\PYZus{}o}\PY{p}{)}\PY{o}{*}\PY{o}{*}\PY{p}{(}\PY{o}{\PYZhy{}}\PY{l+m+mi}{2}\PY{o}{/}\PY{l+m+mi}{7}\PY{p}{)}\PY{p}{)}
        
        \PY{n}{interval\PYZus{}pressure} \PY{o}{=} \PY{n}{np}\PY{o}{.}\PY{n}{linspace}\PY{p}{(}\PY{l+m+mi}{0}\PY{p}{,} \PY{l+m+mi}{1000}\PY{p}{,} \PY{l+m+mi}{1000}\PY{p}{)}
        \PY{n}{Temp\PYZus{}0} \PY{o}{=} \PY{l+m+mf}{2.5} \PY{c+c1}{\PYZsh{}mean temperature of the CMB}
        \PY{n}{A\PYZus{}guess} \PY{o}{=} \PY{l+m+mi}{80000}
        \PY{n}{result\PYZus{}temp} \PY{o}{=} \PY{n}{Temp}\PY{p}{(}\PY{n}{np}\PY{o}{.}\PY{n}{repeat}\PY{p}{(}\PY{n}{Temp\PYZus{}0}\PY{p}{,} \PY{n+nb}{len}\PY{p}{(}\PY{n}{interval\PYZus{}pressure}\PY{p}{)}\PY{p}{)}\PY{p}{,} \PY{n}{np}\PY{o}{.}\PY{n}{repeat}\PY{p}{(}\PY{n}{A\PYZus{}guess}\PY{p}{,} \PY{n+nb}{len}\PY{p}{(}\PY{n}{interval\PYZus{}pressure}\PY{p}{)}\PY{p}{)}\PY{p}{,} \PY{n}{interval\PYZus{}pressure}\PY{p}{)}
        \PY{n}{result\PYZus{}pot\PYZus{}temp1} \PY{o}{=} \PY{n}{pot\PYZus{}Temp}\PY{p}{(}\PY{n}{np}\PY{o}{.}\PY{n}{repeat}\PY{p}{(}\PY{n}{Temp\PYZus{}0}\PY{p}{,} \PY{n+nb}{len}\PY{p}{(}\PY{n}{interval\PYZus{}pressure}\PY{p}{)}\PY{p}{)}\PY{p}{,} \PY{n}{np}\PY{o}{.}\PY{n}{repeat}\PY{p}{(}\PY{n}{A\PYZus{}guess}\PY{p}{,} \PY{n+nb}{len}\PY{p}{(}\PY{n}{interval\PYZus{}pressure}\PY{p}{)}\PY{p}{)}\PY{p}{,} \PY{n}{interval\PYZus{}pressure}\PY{p}{,} \PY{l+m+mi}{1}\PY{p}{)}
        \PY{n}{result\PYZus{}pot\PYZus{}temp2} \PY{o}{=} \PY{n}{pot\PYZus{}Temp}\PY{p}{(}\PY{n}{np}\PY{o}{.}\PY{n}{repeat}\PY{p}{(}\PY{n}{Temp\PYZus{}0}\PY{p}{,} \PY{n+nb}{len}\PY{p}{(}\PY{n}{interval\PYZus{}pressure}\PY{p}{)}\PY{p}{)}\PY{p}{,} \PY{n}{np}\PY{o}{.}\PY{n}{repeat}\PY{p}{(}\PY{n}{A\PYZus{}guess}\PY{p}{,} \PY{n+nb}{len}\PY{p}{(}\PY{n}{interval\PYZus{}pressure}\PY{p}{)}\PY{p}{)}\PY{p}{,} \PY{n}{interval\PYZus{}pressure}\PY{p}{,} \PY{l+m+mi}{10}\PY{p}{)}
        \PY{n}{result\PYZus{}pot\PYZus{}temp3} \PY{o}{=} \PY{n}{pot\PYZus{}Temp}\PY{p}{(}\PY{n}{np}\PY{o}{.}\PY{n}{repeat}\PY{p}{(}\PY{n}{Temp\PYZus{}0}\PY{p}{,} \PY{n+nb}{len}\PY{p}{(}\PY{n}{interval\PYZus{}pressure}\PY{p}{)}\PY{p}{)}\PY{p}{,} \PY{n}{np}\PY{o}{.}\PY{n}{repeat}\PY{p}{(}\PY{n}{A\PYZus{}guess}\PY{p}{,} \PY{n+nb}{len}\PY{p}{(}\PY{n}{interval\PYZus{}pressure}\PY{p}{)}\PY{p}{)}\PY{p}{,} \PY{n}{interval\PYZus{}pressure}\PY{p}{,} \PY{l+m+mi}{100}\PY{p}{)}
        \PY{n}{result\PYZus{}pot\PYZus{}temp4} \PY{o}{=} \PY{n}{pot\PYZus{}Temp}\PY{p}{(}\PY{n}{np}\PY{o}{.}\PY{n}{repeat}\PY{p}{(}\PY{n}{Temp\PYZus{}0}\PY{p}{,} \PY{n+nb}{len}\PY{p}{(}\PY{n}{interval\PYZus{}pressure}\PY{p}{)}\PY{p}{)}\PY{p}{,} \PY{n}{np}\PY{o}{.}\PY{n}{repeat}\PY{p}{(}\PY{n}{A\PYZus{}guess}\PY{p}{,} \PY{n+nb}{len}\PY{p}{(}\PY{n}{interval\PYZus{}pressure}\PY{p}{)}\PY{p}{)}\PY{p}{,} \PY{n}{interval\PYZus{}pressure}\PY{p}{,} \PY{l+m+mi}{500}\PY{p}{)}
        \PY{n}{plt}\PY{o}{.}\PY{n}{plot}\PY{p}{(}\PY{n}{result\PYZus{}temp}\PY{p}{,} \PY{n}{interval\PYZus{}pressure}\PY{p}{,} \PY{n}{label}\PY{o}{=}\PY{l+s+s1}{\PYZsq{}}\PY{l+s+s1}{Temperature}\PY{l+s+s1}{\PYZsq{}}\PY{p}{)}
        \PY{n}{plt}\PY{o}{.}\PY{n}{plot}\PY{p}{(}\PY{n}{result\PYZus{}pot\PYZus{}temp1}\PY{p}{,} \PY{n}{interval\PYZus{}pressure}\PY{p}{,} \PY{l+s+s1}{\PYZsq{}}\PY{l+s+s1}{\PYZhy{}\PYZhy{}}\PY{l+s+s1}{\PYZsq{}}\PY{p}{,} \PY{n}{label}\PY{o}{=}\PY{l+s+sa}{r}\PY{l+s+s1}{\PYZsq{}}\PY{l+s+s1}{\PYZdl{}}\PY{l+s+s1}{\PYZbs{}}\PY{l+s+s1}{theta\PYZdl{} \PYZdl{}p\PYZus{}o\PYZdl{} = 1}\PY{l+s+s1}{\PYZsq{}}\PY{p}{)}
        \PY{n}{plt}\PY{o}{.}\PY{n}{plot}\PY{p}{(}\PY{n}{result\PYZus{}pot\PYZus{}temp2}\PY{p}{,} \PY{n}{interval\PYZus{}pressure}\PY{p}{,} \PY{l+s+s1}{\PYZsq{}}\PY{l+s+s1}{\PYZhy{}\PYZhy{}}\PY{l+s+s1}{\PYZsq{}}\PY{p}{,} \PY{n}{label}\PY{o}{=}\PY{l+s+sa}{r}\PY{l+s+s1}{\PYZsq{}}\PY{l+s+s1}{\PYZdl{}}\PY{l+s+s1}{\PYZbs{}}\PY{l+s+s1}{theta\PYZdl{} \PYZdl{}p\PYZus{}o\PYZdl{} = 10}\PY{l+s+s1}{\PYZsq{}}\PY{p}{)}
        \PY{n}{plt}\PY{o}{.}\PY{n}{plot}\PY{p}{(}\PY{n}{result\PYZus{}pot\PYZus{}temp3}\PY{p}{,} \PY{n}{interval\PYZus{}pressure}\PY{p}{,} \PY{l+s+s1}{\PYZsq{}}\PY{l+s+s1}{\PYZhy{}\PYZhy{}}\PY{l+s+s1}{\PYZsq{}}\PY{p}{,} \PY{n}{label}\PY{o}{=}\PY{l+s+sa}{r}\PY{l+s+s1}{\PYZsq{}}\PY{l+s+s1}{\PYZdl{}}\PY{l+s+s1}{\PYZbs{}}\PY{l+s+s1}{theta\PYZdl{} \PYZdl{}p\PYZus{}o\PYZdl{} = 100}\PY{l+s+s1}{\PYZsq{}}\PY{p}{)}
        \PY{n}{plt}\PY{o}{.}\PY{n}{plot}\PY{p}{(}\PY{n}{result\PYZus{}pot\PYZus{}temp4}\PY{p}{,} \PY{n}{interval\PYZus{}pressure}\PY{p}{,} \PY{l+s+s1}{\PYZsq{}}\PY{l+s+s1}{\PYZhy{}\PYZhy{}}\PY{l+s+s1}{\PYZsq{}}\PY{p}{,} \PY{n}{label}\PY{o}{=}\PY{l+s+sa}{r}\PY{l+s+s1}{\PYZsq{}}\PY{l+s+s1}{\PYZdl{}}\PY{l+s+s1}{\PYZbs{}}\PY{l+s+s1}{theta\PYZdl{} \PYZdl{}p\PYZus{}o\PYZdl{} = 500}\PY{l+s+s1}{\PYZsq{}}\PY{p}{)}
        \PY{n}{plt}\PY{o}{.}\PY{n}{ylabel}\PY{p}{(}\PY{l+s+sa}{r}\PY{l+s+s1}{\PYZsq{}}\PY{l+s+s1}{Units of pressure \PYZdl{}p\PYZdl{}}\PY{l+s+s1}{\PYZsq{}}\PY{p}{)}
        \PY{n}{plt}\PY{o}{.}\PY{n}{xlabel}\PY{p}{(}\PY{l+s+sa}{r}\PY{l+s+s1}{\PYZsq{}}\PY{l+s+s1}{Units of Temperature \PYZdl{}T\PYZus{}}\PY{l+s+si}{\PYZob{}eq\PYZcb{}}\PY{l+s+s1}{\PYZdl{}}\PY{l+s+s1}{\PYZsq{}}\PY{p}{)}
        \PY{n}{plt}\PY{o}{.}\PY{n}{ylim}\PY{p}{(}\PY{l+m+mi}{1000}\PY{p}{,} \PY{l+m+mi}{0}\PY{p}{)}
        \PY{n}{plt}\PY{o}{.}\PY{n}{title}\PY{p}{(}\PY{l+s+s1}{\PYZsq{}}\PY{l+s+s1}{Radiative Equilibrium Temperature as a function of pressure}\PY{l+s+s1}{\PYZsq{}}\PY{p}{)}
        \PY{n}{plt}\PY{o}{.}\PY{n}{grid}\PY{p}{(}\PY{k+kc}{True}\PY{p}{)}
        \PY{n}{plt}\PY{o}{.}\PY{n}{legend}\PY{p}{(}\PY{p}{)}
        \PY{n}{plt}\PY{o}{.}\PY{n}{show}\PY{p}{(}\PY{p}{)}
\end{Verbatim}

    \begin{Verbatim}[commandchars=\\\{\}]
C:\textbackslash{}Users\textbackslash{}Jesus\textbackslash{}Anaconda3\textbackslash{}lib\textbackslash{}site-packages\textbackslash{}ipykernel\_launcher.py:11: RuntimeWarning: divide by zero encountered in power
  \# This is added back by InteractiveShellApp.init\_path()

    \end{Verbatim}

    \begin{center}
    \adjustimage{max size={0.9\linewidth}{0.9\paperheight}}{output_3_1.png}
    \end{center}
    { \hspace*{\fill} \\}
    
    Notice that the temperature profile shown above depicts the temperature
of the parcel increasing as height decreases. The potential temperature
profile for four difference reference pressures \(P_o\) are shown as
dotted lines. An increasing trend is shown along the radiative
equilibirum temperature as pressure increases. Since pressure is
inveresely proportional to height, where minimum pressure of the
atmosphere is reached at a maximum height and maximum pressure is
reached at the minimum height, all potential temperature profiles
illustrate an increasing trend relative to increasing height. This
results in an unstable equilibrium where an airparcel will flow towards
lower pressure along the temperature profile given an initial
perturbation. The instability results in the atmosphere being dominated
by convection which is constrained by the top of the atmosphere. In this
example we look at the case of the effective pressure being 500 Pa. The
red arrows show the perturbation along the temperature profile. The red
arrows show the adaibatic flow of the air parcel after the perturbation
is set. Since the temperature profile of the gas is greater than the
potential temperature profile, and since we mentioned earlier that there
is an increasing trend in temperature and pressure, then the
perturbation makes the air parcel unstable forcing it to flow to the top
of the atmosphere in an adiabatic expansion process. Radiative energy
transfer will only dominate the atmosphere below 500 Pa and convection
will dominate the atmosphere above 500 Pa. Since particle density, and
therefore pressure, is an exponential function dependent on height, then
this implies that the majority of the atmosphere is dominated by
convection. We can prove this by looking into the rate of change of the
potential temperature relatived to pressure:
\[\frac{\partial \theta}{\partial P} = \frac{\partial T}{\partial P}\left( \frac{P}{P_o}\right)^{-\frac{2}{7}} - \frac{2}{7}T \left( \frac{P}{P_o}\right)^{-\frac{2}{7} - 1}(\frac{1}{P_o})\]
\[\frac{\partial \theta}{\partial P} = \theta \left( \frac{1}{T}\frac{\partial T}{\partial P} - \frac{2}{7}(\frac{1}{P}) \right)\]
Note that
\(\frac{\partial T}{\partial P} = \frac{3A T_o P}{4}(1 + \frac{3}{2}AP^2)^{-\frac{3}{4}} = \frac{3A T P}{4(1 + \frac{3}{2}AP^2)}\).
Then:
\[\frac{\partial \theta}{\partial P} = \theta \left( \frac{3A P}{4(1 + \frac{3}{2}AP^2)} - \frac{2}{7}(\frac{1}{P}) \right)\]
\[\frac{\partial \theta}{\partial P} = T_o(1 + \frac{3}{2} AP^2)^{1/4} \left(\frac{P}{P_o}\right)^{\frac{-2}{7}} \left( \frac{3A P}{4(1 + \frac{3}{2}AP^2)} - \frac{2}{7}(\frac{1}{P}) \right)
\] Notice how as P approaches a large magnitude, the rate of change of
the potential energy and therefore the adiabatic rate of the air parcel
reaches a net value of zero. As P approaches a small magnitude, the rate
of change abruptly changes negative and thus implies a decreasing
potential temperature as height increases. This proves that radiative
energy transfer is the dominant source of energy transfer above a
certain critical pressure, and convection is dominant source of energy
transfer below this critical pressure.

    \begin{Verbatim}[commandchars=\\\{\}]
{\color{incolor}In [{\color{incolor} }]:} 
\end{Verbatim}


    % Add a bibliography block to the postdoc
    
    
    
    \end{document}
